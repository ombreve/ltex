% lplain.tex
% Une adaptation du format plain TeX à la langue française :
% - jeu de caractères « lmodern » en codage LY1 (texnansi) ;
% - flux d’entrée utf8 associé à LY1 (via encTeX) ;
% - césures françaises et anglaises ;
% - signes de ponctuation double et guillemets ;
% - majuscules droites en mathématiques ;
% - impression sur feuilles A4.
%

%%
%% Test de encTeX
%%

% Les catcodes minima pour démarrer.
\catcode`\{=1 \catcode`\}=2 \catcode`\#=6 \catcode`\^=7

\ifx\xprncode\undefined
  \message{ fatal error: the '-enc' option is required to process this file.}%
  \batchmode\end
\fi

%%
%% Chargement de plain.tex avec fonts lmodern
%%

% Les 16 fonts de plain en version lmodern (OT1/OML/OMS)
\font\tenrm=rm-lmr10
\font\sevenrm=rm-lmr7
\font\fiverm=rm-lmr5
\font\teni=lmmi10
\font\seveni=lmmi7
\font\fivei=lmmi5
\font\tensy=lmsy10
\font\sevensy=lmsy7
\font\fivesy=lmsy5
\font\tenex=lmex10
\font\tenbf=rm-lmbx10
\font\sevenbf=rm-lmbx7
\font\fivebf=rm-lmbx5
\font\tentt=rm-lmtt10
\font\tensl=rm-lmro10
\font\tenit=rm-lmri10

\def\gobbleline{\begingroup \catcode`\%=12 \catcode13=12 \gulp}
\begingroup
  \lccode1=13 \lowercase{\gdef\gulp#1^^A{\endgroup}}
\endgroup
\let\ofont=\font \let\font=\gobbleline
\input plain
\let\font=\ofont \let\ofont=\undefined
\let\gobbleline=\undefined \let\gulp=\undefined

\catcode`@=11

%%
%% Fonts lmodern en codage LY1 (texnansi)
%%

\let\otenrm=\tenrm
\let\otenbf=\tenbf
\let\otentt=\tentt
\let\otensl=\tensl
\let\otenit=\tenit
\font\tenrm=texnansi-lmr10
\font\tenbf=texnansi-lmbx10
\font\tentt=texnansi-lmtt10
\font\tensl=texnansi-lmro10
\font\tenit=texnansi-lmri10

\chardef\aa             = "E5
\chardef\AA             = "C5
\chardef\bullet         = "94
\chardef\cent           = "A2
\chardef\copyright      = "A9
\chardef\currency       = "A4
\chardef\dag            = "86
\chardef\ddag           = "87
\chardef\degree         = "B0
\chardef\dh             = "F0
\chardef\DH             = "D0
\def\dots               {\relax \ifmmode\ldots \else ^^85\fi}
\chardef\euro           = "01
\chardef\florin         = "83
\chardef\guillemotleft  = "AB
\chardef\guillemotright = "BB
\chardef\guilsinglleft  = "8B
\chardef\guilsinglright = "9B
\chardef\l              = "90
\chardef\L              = "80
\chardef\P              = "B6
\chardef\periodcentered = "B7
\chardef\perthousand    = "89
\chardef\pounds         = "A3
\chardef\quotedblbase   = "84
\chardef\quotedblleft   = "93
\chardef\quotedblright  = "94
\chardef\quotesinglbase = "82
\chardef\quotesingle    = "81
\chardef\registered     = "AE
\chardef\S              = "A7
\chardef\th             = "FE
\chardef\TH             = "DE
\chardef\trademark      = "99
\chardef\yen            = "A5

% La série de \if imbriqué n'est pas optimale, mais les caractères
% accentués sont le plus souvent codés en UTF-8 dans le flux d'entrée.

\def\`#1{%
  \if#1a^^e0\else
  \if#1e^^e8\else
  \if#1i^^ec\else
  \if#1o^^f2\else
  \if#1u^^f9\else
  \if#1A^^c0\else
  \if#1E^^c8\else
  \if#1I^^cc\else
  \if#1O^^d2\else
  \if#1U^^d9\else
  \if#1\i^^ec\else
  {\accent18 #1}%
  \fi\fi\fi\fi\fi\fi\fi\fi\fi\fi\fi}

\def\'#1{%
  \if#1a^^e1\else
  \if#1e^^e9\else
  \if#1i^^ed\else
  \if#1o^^f3\else
  \if#1u^^fa\else
  \if#1y^^fd\else
  \if#1A^^c1\else
  \if#1E^^c9\else
  \if#1I^^cd\else
  \if#1O^^d3\else
  \if#1U^^da\else
  \if#1Y^^dd\else
  \if#1\i^^ed\else
  {\accent19 #1}%
  \fi\fi\fi\fi\fi\fi\fi\fi\fi\fi\fi\fi\fi}

\def\^#1{%
  \if#1a^^e2\else
  \if#1e^^ea\else
  \if#1i^^ee\else
  \if#1o^^f4\else
  \if#1u^^fb\else
  \if#1A^^c2\else
  \if#1E^^ca\else
  \if#1I^^ce\else
  \if#1O^^d4\else
  \if#1U^^db\else
  \if#1\i^^ee\else
  {\accent94 #1}%
  \fi\fi\fi\fi\fi\fi\fi\fi\fi\fi\fi}

\def\~#1{%
  \if#1a^^e3\else
  \if#1n^^f1\else
  \if#1o^^f5\else
  \if#1A^^c3\else
  \if#1N^^d1\else
  \if#1O^^d5\else
  {\accent126 #1}%
  \fi\fi\fi\fi\fi\fi}

\def\"#1{%
  \if#1a^^e4\else
  \if#1e^^eb\else
  \if#1i^^ef\else
  \if#1o^^f6\else
  \if#1u^^fc\else
  \if#1y^^ff\else
  \if#1A^^c4\else
  \if#1E^^cb\else
  \if#1I^^cf\else
  \if#1O^^d6\else
  \if#1U^^dc\else
  \if#1Y^^9f\else
  \if#1\i^^ef\else
  {\accent168 #1}%
  \fi\fi\fi\fi\fi\fi\fi\fi\fi\fi\fi\fi\fi}

\def\v#1{%
  \if#1s^^9a\else
  \if#1z^^9d\else
  \if#1S^^8a\else
  \if#1Z^^8d\else
  {\accent20 #1}%
  \fi\fi\fi\fi}

\def\r#1{%
  \if#1a^^e5\else
  \if#1A^^c5\else
  {\accent23 #1}%
  \fi\fi}

\def\c#1{%
  \if#1c^^e7\else
  \if#1C^^c7\else
  {\setbox\z@\hbox{#1}\ifdim\ht\z@=1ex\accent24 #1%
   \else{\ooalign{\hidewidth\char24\hidewidth\crcr\unhbox\z@}}\fi}%
  \fi\fi}

\def\H#1{{\accent6 #1}}
\def\.#1{{\accent5 #1}}
\def\k#1{\setbox\z@\hbox{#1}\ifdim\ht\z@=1ex\accent7 #1%
  \else{\ooalign{\hidewidth\char7\hidewidth\crcr\unhbox\z@}}\fi}

% Tables lccode et uccode
\lccode"19="19 \lccode"DF="DF % \ss
\def\lowup#1#2{\lccode#1=#1\lccode#2=#1\uccode#2=#2\uccode#1=#2}
\lowup{"1A}{"1D}\lowup{"1B}{"1E}\lowup{"1C}{"1F}
\lowup{"90}{"80}\lowup{"9A}{"8A}\lowup{"9C}{"8C}\lowup{"9D}{"8D}
\lowup{"E0}{"C0}\lowup{"E1}{"C1}\lowup{"E2}{"C2}\lowup{"E3}{"C3}
\lowup{"E4}{"C4}\lowup{"E5}{"C5}\lowup{"E6}{"C6}\lowup{"E7}{"C7}
\lowup{"E8}{"C8}\lowup{"E9}{"C9}\lowup{"EA}{"CA}\lowup{"EB}{"CB}
\lowup{"EC}{"CC}\lowup{"ED}{"CD}\lowup{"EE}{"CE}\lowup{"EF}{"CF}
\lowup{"F0}{"D0}\lowup{"F1}{"D1}\lowup{"F2}{"D2}\lowup{"F3}{"D3}
\lowup{"F4}{"D4}\lowup{"F5}{"D5}\lowup{"F6}{"D6}
\lowup{"F8}{"D8}\lowup{"F9}{"D9}\lowup{"FA}{"DA}\lowup{"FB}{"DB}
\lowup{"FC}{"DC}\lowup{"FD}{"DD}\lowup{"FE}{"DE}\lowup{"FF}{"9F}
\let\lowup=\undefined

%%
%% Flux d'entrée UTF-8
%%

% Séquences UTF-8 des caractères accentués du français
% (àâéèêëîïôùûüÿçæœ) associés à une position LY1.
\mubyte ^^e0 ^^c3^^a0\endmubyte % à
\mubyte ^^c0 ^^c3^^80\endmubyte % À
\mubyte ^^e2 ^^c3^^a2\endmubyte % â
\mubyte ^^c2 ^^c3^^82\endmubyte % Â
\mubyte ^^e9 ^^c3^^a9\endmubyte % é
\mubyte ^^c9 ^^c3^^89\endmubyte % É
\mubyte ^^e8 ^^c3^^a8\endmubyte % è
\mubyte ^^c8 ^^c3^^88\endmubyte % È
\mubyte ^^ea ^^c3^^aa\endmubyte % ê
\mubyte ^^ca ^^c3^^8a\endmubyte % Ê
\mubyte ^^eb ^^c3^^ab\endmubyte % ë
\mubyte ^^cb ^^c3^^8b\endmubyte % Ë
\mubyte ^^ee ^^c3^^ae\endmubyte % î
\mubyte ^^ce ^^c3^^8e\endmubyte % Î
\mubyte ^^ef ^^c3^^af\endmubyte % ï
\mubyte ^^cf ^^c3^^8f\endmubyte % Ï
\mubyte ^^f4 ^^c3^^b4\endmubyte % ô
\mubyte ^^d4 ^^c3^^94\endmubyte % Ô
\mubyte ^^f9 ^^c3^^b9\endmubyte % ù
\mubyte ^^d9 ^^c3^^99\endmubyte % Ù
\mubyte ^^fb ^^c3^^bb\endmubyte % û
\mubyte ^^db ^^c3^^9b\endmubyte % Û
\mubyte ^^fc ^^c3^^bc\endmubyte % ü
\mubyte ^^dc ^^c3^^9c\endmubyte % Ü
\mubyte ^^ff ^^c3^^bf\endmubyte % ÿ
\mubyte ^^9f ^^c3^^9f\endmubyte % Ÿ
\mubyte ^^e7 ^^c3^^a7\endmubyte % ç
\mubyte ^^c7 ^^c3^^87\endmubyte % Ç
\mubyte ^^e6 ^^c3^^a6\endmubyte % æ
\mubyte ^^c6 ^^c3^^86\endmubyte % Æ
\mubyte ^^9c ^^c5^^93\endmubyte % œ
\mubyte ^^8c ^^c5^^92\endmubyte % Œ

% Quelques séquences supplémentaires utiles.
\mubyte '     ^^e2^^80^^99\endmubyte     % ’
\mubyte \dots ^^e2^^80^^a6\endmubyte     % …
\mubyte ~     ^^c2^^a0\endmubyte         % nbsp
\mubyte \og   ^^c2^^ab\endmubyte         % «
\mubyte \fg   ^^c2^^bb\endmubyte         % »
\mubyte \og   ^^c2^^ab^^c2^^a0\endmubyte % « suivi de nbsp
\mubyte \fg   ^^c2^^a0^^c2^^bb\endmubyte % » précédé de nbsp

% Active le décodage du flux d'entrée UTF-8 en LY1.
\mubytein=1

%%
%% Césures
%%

% Césures anglaises (déjà chargées par plain).
\def\english{\language=\z@
  \lefthyphenmin=2 \righthyphenmin=3 \lccode`\'=\z@}

% Césures françaises (le fichier de motifs est codé en utf8).
\newlanguage\lang@french
\begingroup
  \language=\lang@french \lccode`\'=`\'
  \input hyph-fr
\endgroup
\def\french{\language=\lang@french
  \lefthyphenmin=2 \righthyphenmin=2 \lccode`\'=`\'}

% Pas de césures
\newlanguage\lang@nohyphen
\def\nohyphen{\language=\lang@nohyphen}

%%
%% Typographie française
%%

% Ponctuations doubles.
\def\dblpspace{\hskip.5\fontdimen2\font\relax}
\catcode`;=\active
\def;{\ifhmode\ifdim\lastskip>1sp\unskip\penalty\@M\dblpspace\fi\fi\string;}
\catcode`!=\active
\def!{\ifhmode\ifdim\lastskip>1sp\unskip\penalty\@M\dblpspace\fi\fi\string!}
\catcode`?=\active
\def?{\ifhmode\ifdim\lastskip>1sp\unskip\penalty\@M\dblpspace\fi\fi\string?}
\def\colonspace{\space}
\catcode`:=\active
\def:{\ifhmode\ifdim\lastskip>1sp\unskip\penalty\@M\colonspace\fi\fi\string:}

% Guillemets.
\def\guillspace{\hskip.8\fontdimen2\font
  plus .3\fontdimen3\font minus .8\fontdimen4\font\relax}
\def\og{\leavevmode\guillemotleft\penalty\@M\guillspace\ignorespaces}
\def\fg{\ifdim\lastskip>\z@\unskip\fi\penalty\@M\guillspace\guillemotright}

% Petites capitales sans césure pour les noms de famille et autres.
\font\tencsc=texnansi-lmcsc10
\def\bsc#1{\hbox{\tencsc #1}}

%%
%% Mathématiques
%%

% Majuscules droites en mode mathématique.
\begingroup
  \count0="41
  \loop
    \count1=\count0 \advance\count1 by "7000
    \global\mathcode\count0=\count1
  \ifnum\count0 < "5A \advance\count0 by \@ne \repeat
\endgroup

% Quelques jeux de caractères utiles (AMS symbols et Euler fraktur).
\font\tenmsa    = msam10
\font\sevenmsa  = msam7
\font\fivemsa   = msam5
\font\tenmsb    = msbm10
\font\sevenmsb  = msbm7
\font\fivemsb   = msbm5
\font\teneufm   = eufm10
\font\seveneufm = eufm7
\font\fiveeufm  = eufm5
\newfam\msafam  % "8
\newfam\msbfam  % "9
\newfam\eufmfam % "A
\textfont\msafam          = \tenmsa
\scriptfont\msafam        = \sevenmsa
\scriptscriptfont\msafam  = \fivemsa
\textfont\msbfam          = \tenmsb
\scriptfont\msbfam        = \sevenmsb
\scriptscriptfont\msbfam  = \fivemsb
\textfont\eufmfam         = \teneufm
\scriptfont\eufmfam       = \seveneufm
\scriptscriptfont\eufmfam = \fiveeufm

% Caractères mathématiques utiles.
\mathchardef\leqslant     = "3836
\mathchardef\geqslant     = "383E
\mathchardef\subsetneq    = "3928
\mathchardef\supsetneq    = "3929
\mathchardef\N            = "094E
\mathchardef\Z            = "095A
\mathchardef\D            = "0944
\mathchardef\Q            = "0951
\mathchardef\R            = "0952
\mathchardef\C            = "0943

\let\le=\leqslant
\let\ge=\geqslant
\def\bbb{\fam\msbfam}
\def\frak{\fam\eufmfam}

%%
%% Format a4
%%

% On garde le même \hsize que plain, marges horizontales et verticales
% identiques, entête *et* pagination dans les marges.
% \hsize = 6.5in -> marge = (21-16.51)/2 = 2.245cm
% \vsize = 29.7-2*2.245 = 24.956cm (0.1in=0.254cm pour la pagination)
\hoffset       = -0.295cm % 1in - marge
\voffset       = -0.295cm % \hoffset
\vsize         = 25.21cm  % 29.7cm - 2*marge
\dimen\footins = 22.67cm  % \vsize - 1in

% Adapte \m@g (plain.tex, \magnification).
\def\m@g{\mag\count@
  \hsize         = 6.5 true in
  \vsize         = 25.21 true cm
  \dimen\footins = 22.67 true cm
}

%%
%% Final
%%

\rm
\french \frenchspacing
\def\fmtname{lplain}
\edef\fmtversion{0.577 (\fmtversion)}
\dump
