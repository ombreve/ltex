% Test du format frplain

\magnification=\magstephalf
\parindent=0pt
\centerline{\bf Test du format \fmtname, version \fmtversion}
\bigskip

\beginsection Entr\'ee avec syntaxe \TeX

D\`es No\"el o\`u un z\'ephyr ha\"\i{} me v\^et de gla\c cons w\"urmiens
je d\^\i ne d'exquis r\^otis de b\oe uf au kir \`{a} l'a\"y d'\^age
m\^ur \& c\ae tera !

\smallskip
?`But aren't Kafka's Schlo{\ss} and {\AE}sop's {\OE}uvres
often na{\"\i}ve  vis-\`a-vis the d{\ae}monic ph{\oe}nix's official r\^ole
in fluffy souffl\'es?

\smallskip
{\AA}ngel\aa\ \'Erica K\=aren {\L}au\.ra Mar{\'\i}a N\H{a}ta{\l}{\u\i}e
{\O}ctave T\~a{\'\j}a Ur\v{s}ula Yv{\o}nne Z\"azilie

\beginsection Entrée UTF-8

Dès Noël où un zéphyr haï me vêt de glaçons würmiens je dîne d’exquis
rôtis de bœuf au kir à l’aÿ d’âge mûr \& cætera !

\beginsection Césures françaises

Vérifier dans {\tentt test.log} que l'on a bien une ligne avec
césures anglaises :
\smallskip
{\tentt sig-nal con-tainer}
\smallskip
deux lignes avec césures françaises :
\smallskip
{\tentt si-gnal contai-ner \char`\^\char`\^e9v\char`\^\char`\^e9-ne-ment
  al-g\char`\^\char`\^e8bre}
\smallskip
une ligne sans césures :
\smallskip
{\tentt signal container}

{\english \showhyphens{signal container}}
\showhyphens{signal container \'ev\'enement alg\`ebre}
\showhyphens{signal container événement algèbre}
{\nohyphen \showhyphens{signal container}}

\beginsection Ponctuations doubles

Sans espace avant! avec un espace avant ! avec un espace insécable avant !
\par
Sans espace avant? avec un espace avant ? avec un espace insécable avant ?
\par
Sans espace avant; avec un espace avant ; avec un espace insécable avant ;
\par
Et le deux points : fin.

\beginsection Guillemets

\og Avec les macros dédiées \fg, et avec les caractères guillemets\par
«sans espace»,\par
« avec espaces »,\par
« avec espaces insécables ».

\beginsection Mathématiques

$$ A B C \dots a b c $$
$$ \le \ge $$
$$ \N\D\Q\R\C $$

\bye

